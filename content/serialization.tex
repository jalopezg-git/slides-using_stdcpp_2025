% ==== Practical Case: (de-)serialization ====
\section{Practical Case: (de-)serialization}

\begin{frame}[fragile]{Introduction to serialization (1)}
  \textbf{Serialize:} lay out an in-memory \textbf{data structure} as a \textbf{sequence of bytes}.
  \vfill

  \begin{block}{Note that\ldots}
    \begin{itemize}
    \item A type can have nested pointers / references

    \item \inlineCode{sizeof(int)}, \inlineCode{sizeof(long)}, etc. can vary depending on target\footnote{See, e.g. ILP32, LLP64 models.}
      \begin{lstlisting}[style=c++]
        static_cast(sizeof(char) == 1); // ??
        static_cast(sizeof(int) == 4); // ??
      \end{lstlisting}

    \item Differences in machine endianness

    \item Types require alignment (+ maybe padding)
    \end{itemize}
  \end{block}
\end{frame}

\begin{frame}{Introduction to serialization (2)}
  \begin{block}{Can be just \texttt{memcpy()} if\ldots}
    \begin{itemize}
      \itemsep=1em
    \item Data structure is trivially copiable

    \item No pointers / references to other data structures

    \item Uses fixed-width types, e.g. \texttt{uint32\_t}

    \item Machine boundary is never crossed (e.g. IPC)

    \item Alignment is satisfied on de-serialization
    \end{itemize}
  \end{block}

  \begin{onlyenv}<2>
    \overlayLayer{
      {\bfseries Else, implement proper data serialization!}\\[1em]
      \begin{itemize}
      \item Encode numbers using a common endianness, e.g. big-endian
        %
      \item Flatten data structure to byte buffer $\rightarrow$ follow pointers / references
      \end{itemize}
    }
  \end{onlyenv}
\end{frame}

\begin{frame}[fragile]{Example}
  Let's consider this simple data structure
  \begin{lstlisting}[style=c++]
    struct Foobar {
      uint64_t l64;
      uint32_t i32;

      static size_t Serialize(const Foorbar& obj, unsigned char *buf);
      static Foobar Deserialize(const unsigned char *buf);
    };
  \end{lstlisting}
\end{frame}

\begin{frame}[fragile]{Example: machine boundary NOT crossed: '(de-)serialize'}
  \begin{lstlisting}[style=c++]
    size_t Foobar::Serialize(const Foorbar& obj, unsigned char *buf) {
      memcpy(buf, &obj, sizeof(Foobar));
    }

    Foobar Foobar::Deserialize(const unsigned char *buf) {
      Foobar ret;
      memcpy(&ret, buf, sizeof(Foobar));
      return ret;
    }
  \end{lstlisting}

  \begin{onlyenv}<2>
    \overlayLayer{
      Padding bytes are \textbf{unspecified}. \alert{If you care, avoid this!}.
    }
  \end{onlyenv}
\end{frame}

\begin{frame}[fragile]{Example: machine boundary crossed: (de-)serialize}
  \begin{lstlisting}[style=c++]
    size_t Foobar::Serialize(const Foorbar& obj, unsigned char *buf) {
      buf += SerializeUInt64(obj.l64, buf);
      buf += SerializeUInt32(obj.i32, buf);
    }

    Foobar Foobar::Deserialize(const unsigned char *buf) {
      Foobar ret;
      buf += DeserializeUInt64(buf, obj.l64);
      buf += DeserializeUInt32(buf, obj.i32);
      return ret;
    }
  \end{lstlisting}

  \begin{onlyenv}<2>
    \overlayLayer{
      Example of UInt32 little-endian (de-)serialization\\[1ex]
      \lstinputlisting[style=c++]{code/serialize.cpp}
    }
  \end{onlyenv}
\end{frame}

\begin{frame}{Summary}
  \begin{block}{As a rule of thumb}
    \begin{itemize}
      \itemsep=1em
    \item \textit{standard-layout / POD} + machine boundary not crossed (ever) $\rightarrow$ copy + start lifetime on receiving end

    \item Rest: implement proper serialization!
    \end{itemize}
  \end{block}

  \vfill
  See also: \linkButton{https://www.boost.org/doc/libs/1_87_0/libs/serialization/doc/index.html}{Boost Serialization}
\end{frame}
